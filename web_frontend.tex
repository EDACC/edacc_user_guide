\upshape
\index{Web Frontend}
\subsection{Introduction}
The Web Frontend provides access to experiment information and analysis tools in a read-only manner
and accessible by a web browser.

\subsection{System requirements}
\index{System requirements, Web Frontend}
The web frontend is implemented as Python WSGI web application and makes use of several libraries.
Since it interfaces with R to draw plots it also depends on R and a Python interface to R, which unfortunately
only works properly on Linux right now.
WSGI applications can be deployed on a variety of web servers or even run standalone on a web server that comes with the
Python standard library.
The following list contains all dependencies and prerequisites of the web frontend (see below for installation instructions).
\begin{itemize}
\item Python 2.6.5 or 2.7 http://www.python.org
\item R 2.11 (language for statistical computing and graphics)
\item R package 'np' (available via R's CRAN)
\item SQLAlchemy 0.6.5 (SQL Toolkit and Object Relational Mapper)
\item mysql-python 1.2.3c1 (Python MySQL adapter)
\item Flask 0.6 (Micro Webframework)
\item Flask-WTF 0.3.3 (Flask extension for WTForms)
\item Flask-Actions 0.5.2 (Flask extension)
\item Werkzeug 0.6.2 (Webframework, Flask dependency)
\item Jinja2 2.5 (Template Engine)
\item PyLZMA 0.4.2 (Python LZMA SDK bindings)
\item rpy2 2.1.4 (Python R interface)
\item PIL 1.1.7 (Python Imaging Library)
\item Numpy 1.5.1
\item pygame 1.9 (Graphics library)
\end{itemize}

\subsection{Installation}
\index{Installation, Web Frontend}
To get rpy2 working the GNU linker (ld) has to be able to find libR.so. Add the folder containing
libR.so (usually /usr/lib/R/lib) to the ld config: Create a file called R.conf containing the
path in the folder /etc/ld.so.conf.d/ and run ldconfig without parameters as root to update.
Additionally, you have to install the R package 'np' which provides non-parametric statistical
methods. This package can be installed by running "install.packages('np')" within the R interpreter (as root).

The following installation example outlines the step that have to be taken to install the web frontend on Ubuntu 10.04
running on the Apache 2.2.14 web server. For performance reasons (e.g. query latency) the web frontend should run on the
same machine that the EDACC database runs on.
\marginlabel{\Eexample}
\begin{enumerate}
\item Install Apache and the WSGI module: \begin{verbatim}apt-get install apache2 libapache2-mod-wsgi\end{verbatim}
\item{ Copy the web frontend files to /srv/edacc\_web/, create an empty error.log file and change their ownership to the Apache user: 
\begin{verbatim}
  touch /srv/edacc_web/error.log
  chown www-data:www-data -R /srv/edacc_web
\end{verbatim}
}
\item{ Create an Apache virtual host\\
(new file at /etc/apache2/sites-available/edacc\_web)
\begin{verbatim}
<VirtualHost *:80>
  ServerAdmin email@email.com
  ServerName foo.server.com

  LimitRequestLine 51200000

  WSGIDaemonProcess edacc processes=1 threads=15
  WSGIScriptAlias / /srv/edacc_web/edacc_web.wsgi

  Alias /static/ /srv/edacc_web/edacc/static/

  <Directory /srv/edacc_web>
    WSGIProcessGroup edacc
    WSGIApplicationGroup %{GLOBAL}
    Order deny,allow
    Allow from all
  </Directory>

  <Directory /srv/edacc_web/edacc/static>
    Order allow,deny
    Allow from all
  </Directory>
</VirtualHost>
\end{verbatim}
}
\item{Install dependencies and create a virtual environment for Python libraries:
\begin{verbatim}
apt-get install python-pip python-virtualenv python-scipy python-pygame python-imaging
virtualenv /srv/edacc_web/env
apt-get build-dep python-mysqldb
apt-get install r-base
echo "/usr/lib/R/lib" > /etc/ld.so.conf.d/R.config
ldconfig
source /srv/edacc_web/env/bin/activate
pip install mysql-python
pip install rpy2
pip install flask flask-wtf flask-actions
pip install sqlalchemy pylzma numpy
\end{verbatim}
}
\item{Install R libraries (''R'' launches the R interpreter):
\begin{verbatim}
R
install.packages('np')
\end{verbatim}
}
\item{Create a WSGI file at /srv/edacc\_web/edacc\_web.wsgi with the following contents:
\begin{verbatim}
import site, sys, os
site.addsitedir('/srv/edacc_web/env/lib/python2.6/site-packages')
sys.path.append('/srv/edacc_web')
sys.path.append('/srv/edacc_web/edacc')
os.environ['PYTHON_EGG_CACHE'] = '/tmp'
sys.stdout = sys.stderr
from edacc.web import app as application
\end{verbatim}
}
\item Configure the web frontend by editing /srv/edacc\_web/edacc/config.py, see~\ref{wf:configuration} for details.
\item{Enable the Apache virtual host created earlier:
\begin{verbatim}
a2ensite edacc_web
service apache2 restart
\end{verbatim}
}
\item The web frontend should now be running under http://foo.server.com/
\end{enumerate}

\subsection{Configuration}
\label{wf:configuration}
\index{Configuration, Web Frontend}
All configuration is done in a Python file located at ''edacc/config.py''. The options are documented in the sample configuration
file which is included in the distribution package.\attention Most importantly, you should disable debugging mode when making the Web Frontend
accessible from the network to avoid security problems. At the end of the file you can configure the database connection and the list
of EDACC databases that should be made available by the Web Frontend.

\subsection{Feature overview}