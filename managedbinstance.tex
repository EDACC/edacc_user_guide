% !TeX root = managedbmode.tex
\subsubsection{Instances}
\marginlabel{Instance} An instance is a practical instantiation of a problem. The instances tab provides functions for the user to add, remove, generate and organize instances.

\marginlabel{Instance Class} Instance classes enables the user to group and organize instances into different categories. It's possible that an instance is assigned to several instance classes. An instance classes can include other instance classes and are represented as an tree.

\marginlabel{Add Instance} To add one or more instances, use the button Add. In the following dialogue are four possible choices. 
\begin{enumerate}
	\item If automatic class generation is selected, the added instances are added to instance classes which
	are generated from the dependent on the directory of the instances to add. .
	
	\item If the automatic class generation isn't selected, the user have to choose an instance class from the 		instance class table of the dialogue. Else if automatic class generation is selected, the choice of an 			instance class is optional.
	
	\item To save the instances as compressed files in the database, select Compress.
	
	\item In the field File Extension, the user has to define the file extension of the instance files to add.
\end{enumerate}

To continue the process, the user has to use the button Ok and select the directory of the instances or their explicit files. This depends on the decisions made in the previous dialogue.
 

\marginlabel{Remove Instance}

\marginlabel{Generate Instance}

\marginlabel{Export Instance}

\marginlabel{Compute Property} 