% !TeX root = managedbmode.tex
\subsubsection{Instances}
\marginlabel{Instance} An instance is a practical instantiation of a problem. The instances tab provides functions for the user to add, remove, generate and organize instances.

\marginlabel{Instance Class} Instance classes enables the user to group and organize instances into different categories. It's possible that an instance is assigned to several instance classes. An instance classes can include other instance classes and are represented as an tree.

\marginlabel{Add Instance} To add one or more instances, use the button Add. In the following dialogue are four possible choices. 
\begin{enumerate}
	\item If automatic class generation is selected, the added instances are added to instance classes which
	are generated from the dependent on the directory of the instances to add. .
	
	\item If the automatic class generation isn't selected, the user have to choose an instance class from the 		instance class table of the dialogue. Else if automatic class generation is selected, the choice of an 			instance class is optional.
	
	\item To save the instances as compressed files in the database, select Compress.
	
	\item In the field File Extension, the user has to define the file extension of the instance files to add.
\end{enumerate}

To continue the process, the user has to use the button Ok and select the directory of the instances or their explicit files. This depends on the decisions made in the previous dialogue.
 

\marginlabel{Remove Instance} Use the button Remove under the instance table, to remove instances from the database.

\marginlabel{Generate Instance} ?

\marginlabel{Export Instance} The export function of instances from EDACC is provided by the Button Export, located on the left side under the instance table. The user has to choose the directory, in which the instances are exported.

\marginlabel{Compute Property} To compute a property of a group of instances, the user has to select the instances in the table and use the button Compute Property. After pressing the button a new dialogue is shown, with the possible properties to compute. To start the computation process, a property has to be chosen and the button Compute to be pressed.

\marginlabel{Filter Instances} The user can call the filter function dialogue of the instance table by using the button Filter. ....

\marginlabel{Select Columns of Instances} A selection of columns of the instance table can be called by pressing the button Select Columns. The appearing dialogue shows two kinds of columns which can be selected by the user, the Basic Columns and the Instance Property Columns. The variety of property columns depends on the number of defined instance properties.  

\marginlabel{Add Instance to Instance Class} The user has to select a group of instances, use the button Add to Class and select the instance class to which the instances have to be added to in the appeared dialogue.

\marginlabel{Remove Instance from Instance Class} To remove a group of instances from a instance class, the user has to select these instances and press the button Remove from Class.
