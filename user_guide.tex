\documentclass[twoside,a4paper]{refart}

\usepackage{makeidx}
\usepackage{glossary}
\makeglossary
\usepackage{graphicx}
\usepackage{array}
\usepackage[unicode,a4paper]{hyperref} 
%\usepackage[german]{babel}
\usepackage[latin1]{inputenc}
\usepackage{amsmath}
\usepackage{amsthm}
\usepackage{color}
\usepackage{comment}
%\usepackage[all,draft]{draftcopy}
\usepackage{graphicx}
\usepackage{type1cm}
\usepackage{eso-pic}
\usepackage{color}
\usepackage{xspace}

\makeatletter
%\AddToShipoutPicture{%
%            \setlength{\@tempdimb}{.5\paperwidth}%
%            \setlength{\@tempdimc}{.5\paperheight}%
%            \setlength{\unitlength}{1pt}%
%            \put(\strip@pt\@tempdimb,\strip@pt\@tempdimc){%
%        \makebox(0,0){\rotatebox{45}{\textcolor[gray]{0.85}%
%        {\fontsize{6cm}{6cm}\selectfont{DRAFT}}}}%
%            }%
%}
\makeatother



\definecolor{edaccblue}{rgb}{0,0.2,0.6}

\definecolor{green}{rgb}{0,0.8,0.2}
\hypersetup{pdftex=true, colorlinks=true, breaklinks=true, linkcolor=edaccblue, menucolor=edaccblue,pagecolor=edaccblue, urlcolor=edaccblue}


\newtheoremstyle{dotless} % Name
                        {0.5em}    % Space above
                        {0.5em}    % Space below
                        {}         % Body font
                        {}         % Indent amount
                        {\bfseries}% Theorem head font
                        {:}        % Punctuation after theorem head
                        {\newline} % Space after theorem head
                        {}         % Theorem head spec (can be left empty, meaning 'normal')

\theoremstyle{dotless}
\newtheorem{definition}{Definition}[section]

\newcommand{\ie}{i.\,e.,}
\newcommand{\eg}{e.\,g..}
\newcommand{\edacc}{\textbf{\color{edaccblue}EDACC}\xspace}
\newcommand{\mysql}{MySQL\textsuperscript{\texttrademark}}
\newcommand{\todo}{\textbf{\color{red}TODO:}}
\newcommand{\ml}[1]{\marginlabel{#1}\index{#1}}


\newcounter{ex}
\setcounter{ex}{1}
\newcommand{\Eexample}{\color{green}Example \arabic{ex}:  \addtocounter{ex}{1}}

\title{
\begin{tabular}{>{\raggedright}m{4cm}>{\raggedleft}m{10cm}}
%\begin{tabular}{|l|r|}
EDACC \\User Guide \\version 0.1\\ & \includegraphics[scale=0.3]{edacclogo.jpg}
\end{tabular}
}

\author{Copyright\copyright by  Adrian Balint, Daniel Diepold, Daniel Gall, Simon Gerber, Gregor Kapler, Robert Retz, Melanie Handel}
\date{}


\pagestyle{myfootings}
\markboth{EDACC User Guide}%
         {EDACC User Guide}
\makeindex 

\setcounter{tocdepth}{2}

\begin{document}

\maketitle



\begin{abstract}
        We present the main capabilities of EDACC and describe how to use EDACC for managing solvers and instances, create experiments with them, launch them on different computer clusters, monitor them and then analyze the results. 
\end{abstract}



\tableofcontents

\newpage


%%%%%%%%%%%%%%%%%%%%%%%%%%%%%%%%%%%%%%%%%%%%%%%%%%%%%%%%%%%%%%%%%%%%
\begin{comment}
\section*{User guide zum User Guide}
\color{red} Bevor Ihr etwas in diesem user guide was reinschreiben wollt/sollt bitte diesen Kapitel durchlesen.
\color{black}
Alle Autoren sollten versuchen die folgenden Richlinien zu folgen, sodass die Arbeit ein homogenes Erscheinen bekommt auch wenn viele Autoren dran arbeiten. 
\subsection{Darstellungskonventionen}
Das ist ein User Guide, folglich sollten die wichtigen Informationen f�r den Benutzer sehr schnell auffindbar sein. 
Dazu verhelfen folgende Konzepte:

	\marginlabel{Index} Indexierung aller wichtigen W�rter; insbesondere Schl�sselw�rter. Der Index und die pdf-hyperreferenzen sollen ein schnelles suchen erm�glichen. Setzen von index w�rter mit \verb|\index{wort}|.
	
	Um wichtige Dinge schnell zu finden sollen auch seitliche Hinweise helfen. 
	\marginlabel {seitlicher Hinweis auf} diese werden mit\\ \verb|\marginlabel{schl�sselwort}| gesetzt. 
	
	 Auf Beispiele weist man am besten mit \verb|\marginlabel{\Eexample}| hin. \marginlabel{\Eexample}
	
	\marginlabel{Verweise} Wenn man auf eine andere stelle im Text refenrenzueren will, was man so oft wie m�glich machen sollte dann erscheint die Referenz auch im linken Rand mit: \verb|\seealso{label}|. Wie zum Beispiel: Ein �berblick �ber diesen user-guide gibts in Kapitel \ref{outline} \seealso{Section \ref{outline}}.
	
	\marginlabel{Hinweise} Falls etwas doch sehr wichtig sein sollt, im Regel Besonderheiten auf die man achten sollte so kann man den Leser mit \attention \verb|\attention| darauf aufmerksam machen!
	
	\marginlabel{Glosareintr�ge} Wichtige Terme sollten am besten auch eine Definition haben, oder wenigstens eine Erkl�rung was damit gemeint ist. Das kann man am besten mit 
	\verb|\glossary{name={Glossareintrag}, description={Beschreibung}}| \glossary{name={Glossareintrag},description={Das ist ein Eintrag in dem Glossar des Dokumnets!}}. 
	
\subsection{Inhaltliche Konventionen}	
\marginlabel{Audienz identifizeren} Ich gehe davon aus, dass die meisten Leute die EDACC verwenden werden Informatiker sein werden, oder eine Unterart davon. Folglich k�nnen wir davon ausgehen, dass sie mit den g�ngisten Begriffe vertraut sind (Ein Glossareintrag zu diesen w�rde tortzdem nicht schaden). Der Benutzer wird immer als \textbf{user} im Text angesprochen. 
\marginlabel{Verwendungsart des user-guides} Ich gehe auch davon aus, dass die meisten users diese Hilfe als Nachschlagewerk verwenden werden. Folglich soll auch der Inhalt task-orientiert sein. Das bedeutet dass man nicht das System an sich versucht zu beschreiben sondern die Aufgaben beschreibt und dadurch eher die Systembeschreibung entsteht. Es ist ungef�hr das Gegenteil von dem Paper, wo nur die abstrakten Konzepte des Systems beschrieben worden sind. Also wenn Eure Doku zu sehr nach paper klingt seid ihr auf dem falschen Weg.

\marginlabel{System definieren} Das System wenn es als Ganzes erw�hnt wird sollte mit \textbf{EDACC} erw�hnt werden. Allerdings sollte man vermeiden das Obersystem in Beschreibungen zu verwenden. Besser ist es wenn man sich immer auf dessen Teilkomponenten bezieht: \textbf{DB, GUI, client, WF}. \textbf{Monitor} wird nur als ein Teil des \textbf{WF} betrachtet.

\marginlabel{Workflow angelehnte Beschreibung} Die Beschreibung der einzelenen Aufgaben sollte an dem typischen Workflow von EDACC orientiert sein. 
Die allgeinen Konzepte und Probleme werden in der Anleitung sehr gut beschrieben sein. Im Hauptteil des user-guides sollen die Aufgaben beschrieben werden die man mit EDACC l�sen kann. \marginlabel{Task-Beschreibung}Dabei achtet man auf:
\begin{enumerate}
	\item Identifizierung der Aufgaben (z.B:''Solver hinzuf�gen'')
	\item Aufteilung der Aufgabe in Unteraufgaben: (z.B: ``Solver binaries verwalten und die Parameter spezifizieren'')
	\item Jede Aufgabe wird in Schritten beschrieben die durchnummeriert werden. (z.B: 1. name des solvers eintragen 2. Author eintragen 3. Version (name und verison m�ssen eindeutig sein))
	\item Fallunterscheidungen sollen explizit beschrieben werden wenn der Benutzer eine Entscheidung treffen muss (if-then-else Formulierungen). z.B: (Falls ein Parameter als Boolean markiert wird so ...) 
	\item Reichlich refenrenzieren und  verweisen auf Glossar, Index und andere Stellen im Text wo  (verwandte / ben�tigte) ( Aufgaben / Konzepte ) beschrieben werden.
\end{enumerate}

\end{comment}

	

\section{Outline}
\label{outline}
Here we will have an overview of this user guide specifying where the user can find what!

\section{Introduction}

\subsection{General Terms}
To keep this user-guide consistent we would like to define a couple of terms that will be often used through this document. Even if you are familiar with these, we recommend you to take a short look at them.


\marginlabel{Algorithm} We define an \textbf{algorithm} as an arbitrary computation method.  Examples of well known algorithms are the family of sorting algorithms like bubble-sort, quick-sort or merge-sort.
\marginlabel{Solver} The concrete implementation of an algorithm in an arbitrary programming language is called a \textbf{solver}, which normally has an input and an output. 

A solver is designed to solve a certain type of problem.\marginlabel{Instance} One concrete problem (an instantiation of it) is called a (problem) \textbf{instance} . For the sorting algorithms an example of an instance would be a file containing a sequence of number that has to be sorted.

\marginlabel{Solver Parameters} To control the behavior of a solver it can have parameters which we will call \textbf{solver parameters}. These parameters can be also seen as an input of the solver which is normally passed through the command line. For example the quick-sort algorithm could have a parameter ``pivot'' that can take the values $\{left,right,random\}$. With the help of this parameter the behavior of the solver can be controlled regarding how it should choose the pivot element during sorting. 

\marginlabel{Solver Configuration} A solver together with a fixed set of values for its parameters is called a \textbf{solver configuration}. Randomized quick-sort would be a solver configuration of the quick-sort solver with the parameter ``pivot'' set to $random$. 

\marginlabel{Computing System} To see how a solver performs on a certain instance we need to execute that solver. For this task we need a \textbf{computing system} which in \edacc ca be a single computer, computer cluster of even a grid. 

As \edacc provides a wide variety of statistical analysis tools we need a way to point out different forms of informations.  \marginlabel{Instance Property} We define a \textbf{instance property} as any kind of information that can be extracted from an instance. \marginlabel{Result Property} The output of a solver is called the \textbf{result} and any information that can be computed from the result is called \textbf{result property}. 


\subsection{Motivation}


\index{Algorithm engineering}\marginlabel{Algorithm engineering:}
When designing and implementing algorithms one is at the end of the process confronted with the problem of evaluating the implementation on the targeted problem set. As the authors of \edacc are familiar with algorithms for the satisfiability problem we will take this sort of algorithms as further examples. After designing and implementing a SAT solver we would like to see how it performs on a set of instance problems (let us suppose that our solver is an implementation of a stochastic one \ie the result of the solver on the same instance will be a random variable).

Normally we would start our solver on each instance and record the runtime or some quality measure. This is a sequential process and could be easily performed with the help of simple shell script. But there are some questions that have to be answered before starting the evaluation process. 

\begin{enumerate}
\item How long is the solver allowed to compute on one instance? And how do we restrict that?
\item In the case of randomized solvers, how often do we call the solver on each problem set?
\item Do we limit the resources used by the solver (\ie maximum of memory, maximum stack size)?
\end{enumerate}

Let us now suppose we would like to test our SAT-solver on 100 instances where we allow a timeout of 200 seconds. \marginlabel{\Eexample}  Because of the stochastic nature of the solver we are going to run it for 100 times on each instances. We are not going to limit other resources. Now we get a set of (100 instances) $\times$ (100 runs) that produces a set of 10000 jobs. Having a timeout limit of 200 seconds our computation could take up to $10000\cdot 200=2000000sec\cong24 days$ on a single CPU machine in worst case. 

Now everybody has access to multi-core machines or even some clusters with multiple CPU's. So we could speed up the computation by using this sort of resources but then we get the problem of equally spreading our jobs. And more than that we have to collect the results after that and process them with some statistical tools. 

Most of the researchers solve this problems by writing a collection of scripts. This solution is error-prone and time consuming because there is no very simple way to equally spread jobs across multiple machines. Collecting the results and merging them together can also yield a not negligible amount of work. 
One more disadvantage is that the results can be seldom reproduced without having the complete set of scripts and even then there might be some steps that are not incorporated within the scripts. 

To solve this problems we have designed \edacc. The main goal of \edacc are to: \marginlabel{\edacc features}
\begin{enumerate}
	\item manage solvers and instances and archiving them in a database with the help of a GUI
	\item create experiment settings by configuring solvers and selecting the instances
	\item evaluating the jobs of an experiment on arbitrary many machines
	\item provide analysis tools for the results
	\item provide an online tool to monitor and analyze experiments
\end{enumerate}

\subsection{EDACC Components}

The four major components of \edacc are the:
\begin{enumerate}
	\item Grapical user interface (GUI) 
	\item Database (DB)
	\item Compute client (client) 
	\item Web frontend (WF) (optional)
\end{enumerate}

\subsection{System Requirements}
\begin{enumerate}
	\item \attention GUI - Sun Java 6 (JRE 6), optional: R (see Experiment Mode - README.txt for more details)
	\item \attention {Database - MySQL version 5.1 or above, tested with version 5.1.41 on Ubuntu.
	    The machine the database runs on is the most important factor of the
	    performance of \edacc. The following components will have the greatest
	    impact on database performance:
	    \begin{itemize}
	        \item {The more RAM MySQL can use, the less it has to access slow hard
	            disks on read-transactions. It also enables MySQL to keep indexes and whole tables in memory.
	            This will greatly affect the ability to
	            work on multiple experiments at the same time.}
	        \item {Hard disk performance is not as important as RAM but all data has
	            to be written to the disk eventually which is when fast access time
	            and write throughput become important.}
	        \item {A fast multi-core CPU will enable MySQL to handle more requests concurrently but is
	            not as important as RAM.}
	    \end{itemize}
	    Network latency and bandwidth should also be considered when the GUI and clients
	    are run on remote machines. The clients will write the output of solvers
	    and metadata back to the database so the required bandwidth depends
	    on the size of the generated output and metadata.
	}
	\item Client - see section \ref{clientSR} \attention
	\item Web Frontend - see section \ref{wfSR} \attention
\end{enumerate}


\subsection{Getting started}
To use \edacc you will have to follow these steps: 
\begin{enumerate}
	\item set up a mysql database
	\item download the latest \edacc GUI from sourceforge.org (eventually check for updates within \edacc)
	\item check if the client runs on your target computing system (eventually recompile the client on the targeted computation system)
\end{enumerate}
	
\subsection{MySQL Installation and Setup}
MySQL installation is simple on most Linux distributions. On Ubuntu, for example, you
have to type \texttt{apt-get install mysql-server} and set a root account password
when the installation procedure asks you to.
\marginlabel{MySQL configuration} After installation there are a few settings
that have to be adjusted in order to use MySQL with \edacc. These can be found in the configuration file
my.cnf usually located at \texttt{/etc/mysql/my.cnf}. Adjust the following settings:
\begin{verbatim}
[mysqld] # look for this section
# listen on all IPs/allow network connections :
bind-address = 0.0.0.0
 # maximum packet size (important for large instances):
max_allowed_packet = 2048M

# comment out the skip-networking directive,
# if present:
#skip-networking

# performance related settings
# innodb_buffer_pool_size is the most important parameter
# set this to as much RAM as you can spare on the machine:
innodb_buffer_pool_size = 1024M
\end{verbatim}
After saving the modifications, restart your MySQL server (Ubuntu: \texttt{service mysql restart})
and open a MySQL client session by typing \texttt{mysql -uroot -p} which will then ask you for
the root password you specified during MySQL installation.
\marginlabel{Creating databases}
In the MySQL client shell you can then create an empty database that can be used as \edacc~database
by running the following commands:
\begin{verbatim}
CREATE DATABASE EDACCDB;
GRANT ALL PRIVILEGES ON EDACCDB.* TO 'dbuser'@'%'
  IDENTIFIED BY 'dbuserpassword' WITH GRANT OPTION;
\end{verbatim}
This will create an empty database called \textit{EDACCDB} and grant the MySQL user \textit{dbuser}
with the password \textit{dbuserpassword} all necessary rights. In the \edacc~GUI, client and Web Frontend
you can then use this account when connecting to the database.


\section{Graphical User Interface}
\subsection{Database connection}
Every time you will start \edacc you will be prompted to provide the connection data to the \mysql database you would like to work with. 
\begin{center}
	\includegraphics[width=1.00\textwidth]{snapshots/DBlogin.jpg}
\end{center}

\marginlabel{Host name / IP}In the connection dialog you have to provide the host name or the IP-address of your \mysql database. 

\marginlabel{Port}If you have configured the \mysql server to use an other port then the default \mysql port 3306, you can specify this in the \textbf{Port:} text field. 

\marginlabel{DB name} Further you also have to provide a valid database name and a user along with the corresponding password. 

\marginlabel{Save password} If you would like to save the password of this connections for further usage you can check the \textbf{Save password} check-box. \edacc will save the password for you in a configuration file. \attention The password is saved in plain text, so if other users have access to your private files they will be able to read the password from the configurtion file. 

\marginlabel{Max Connections} \edacc is a multi-threaded program and will use more than on connection to the database to speed up certain tasks. We recommend to allow up to 8 simultaneous database connections, but if you have restrictions on this number you can specify it in the \textbf{Max Connection:} text field. \todo Simon: wenn das weniger sind?

\marginlabel{SSL connections} If you are going to use \edacc to store trusted data we strongly recommend to enable a SSL connection by checking the \textbf{secured connection} check box. \todo Be aware that this kind of connection is only possible is the \mysql server is configured accordingly. 

\marginlabel{Compressed Connections} When working with \edacc through a slow network connection you might want to turn on compression by checking the \textbf{compress connection} check box. 

\marginlabel{Connect} After providing all the information you can connect to the \mysql server. 

\marginlabel{Create DB} When you connect the first time to a database \edacc will create for you all the needed tables. \todo Was passiert wenn man sich an einer falschen DB verbindet, die ein anderes Modell hat. Kann man das EDACC DB Modell neu erzeugen �ber einen bestehenden? 

\marginlabel{DB Model version} 
As \edacc is under full development and the database model may be extended to support new features, \edacc will check if the database model is compatible with the GUI version. Within this check we  differentiate between two cases:
\begin{enumerate}
	\item \ml{DB Model upgrade}The database model version is to old for the GUI. In this case \edacc will offer you the possibility to upgrade your database scheme to the latest version. \todo Werden da auch mehrere versionen �bersprungen? Sprich von version 0.1 auf 0.4?
	\item \ml{GUI update}The database model version is to new for the GUI. In this case you should update the GUI. You can do this by using the automatic update function of \edacc, which can be found under \textbf{Help $\rightarrow$ Check for Updates}. Another possibility is to download the latest release form the project site at \url{http://sourceforge.net/projects/edacc/}. 
\end{enumerate}
\subsection{Modes}
\edacc is split up in two modes:
\begin{enumerate}
	\item Manage DB Mode
	\item Experiment Mode
\end{enumerate}
There is a strict split-up between these two modes. You can be only in one mode when working with \edacc.
When starting \edacc you will always be in the the manage DB mode, which will allow you to manage your solvers and instances before creating experiments with them. 
To switch between modes you have to choose the desired mode from the menu bar \textbf{Mode}. 


% !TeX root = user_guide.tex
\subsection{Manage Database Mode}
% !TeX root = managedbmode.tex
\marginlabel{Solvers} A solver is a program which implements an algorithm for solving a problem. In EDACC a solver is represented by the following information:
\begin{enumerate}
 \item[Name] A human-readable name of the solver.
 \item[Version] The version number of the solver. The combination of name and version must be unique.
 \item[Description] A short description of the solver.
 \item[Authors] The list of the authors of a solver.
 \item[Code] The sourcecode of the solver.
 \item[Several Binaries] A solver can consist of different binaries, which have the same source code but differ in the compile options (eg. the architecture) or the chosen compiler version. There must be at least one solver binary.
\end{enumerate}

\marginlabel{Parameters} Every solver has a list of several parameters which control its behaviour. To build a valid parameter list string, EDACC needs the following information:
\begin{enumerate}
 \item[name] The human-readable internal name of the parameter. This name has no effect to the generated command-line and is only needed for reasons of indentification in the EDACC system.
 \item[prefix] The parameter prefix defines how the parameter is called on the command-line. The Unix program \verb|ls| for example has a parameter with the prefix \verb|-l|.
 \item[Boolean] Some parameters don't have an actual value but act as switches for a certain functionality of a solver. The \verb|-l| parameter of the Unix program \verb|ls| for example is such a boolean parameter.
 \item[Mandatory] Some parameters need to be specified to start the solver binary. Such parameters are called mandatory.
 \item[Space]
 \item[Order] Some solvers need a special order of the parameters. This order is specified by an ascending number. The parameter with the smallest number will be used first in the command-line string. If two parameters have the same order number, the order between those two parameters doesn't matter.
\end{enumerate}

\marginlabel{Add Solver}
By clicking the button ``New'' in the solver panel a new empty line in the solver table is created. To fill the new entry with information fill in the form below the table with the static information of the solver. Optionally you can attach the code of the solver to the entry by clicking on ``Add Code'' and choosing the files or directories from your file system. 

\attention To create a valid solver entry, it is necessary to specify at least one solver binary.

\marginlabel{Add Solver Binary}
The table below the text fields with the static solver information shows the solver binaries which are already attached to the chosen solver. To add another binary, click on the ``Add'' button below the table with the binaries. Choose the binary files which are needed to run the solver from your file system. EDACC then tries to zip the chosen files. This can take a few seconds. 

To complete the process, some information on the binary have to be given:
\begin{itemize}
 \item[Alternative Binary Name] A human-readable name of the binary. This information is only needed that the binary can be recognized by the user in the program.
 \item[Execution File] The main file of the binary, which will be called by the EDACC client to start the binary. You can choose it from the list of the previously chosen binary files. For default, the first file is chosen.
 \item[Additional run command] Some binaries or scripts need a special command to start them (this is very usual for interpreted languages or scripts). For example a Java JAR archive can be started by the additional run command \verb|java -jar|. A preview of the command executed on the grid by the client is shown in the text line below the text field for the addtional run command.
 \item[Version] The version string specifies for example the architecture of the compiled binary or the used compiler. The version of the underlying source code is specified in the solver information, which is described above!
\end{itemize}

Click on ``Add binary'' to complete the process.

\attention All modifications on solvers, solver binaries or parameters are not directly saved to the database. To persist your changes, you can choose the button ``Save To DB''.

\marginlabel{Edit Solver} 
To edit the information of a certain solver, choose the solver from the solver table. The text fields below the table will show the currently saved information of the solver. By changing those values, the information in the solver table will be adjusted automatically.

\marginlabel{Edit Solver Binary}

\marginlabel{Delete Solver Binary} 
To delete a solver binary, choose it in the list of binaries and click on the ``Delete''-Button below the table.
\attention After confirming the delete action, the solver binary will be removed directly from the database!

\marginlabel{Delete Solver}
If you want to delete a solver with all attached information, code, binaries and parameter, click on the ``Delete''-Button in the solver panel. The solver will be removed directly from the database, after confirming the delete action. To delete multiple solvers at once, just hold \verb|Ctrl| in the solver table.

\marginlabel{Add Parameter}
To add a parameter to a solver, choose the solver from the solver table. On the parameters panel, the list of parameters will show all parameters of the chosen solver. By clicking on ``New'' in the parameter panel, a new empty line will appear in the parameters table and is selected automatically. The text fields and checkboxes below the tab show the default values created for the new parameter. To change them, simply change the values in those control fields. The information in the table will adjust automatically. For your comfort, the order value will be incremented automatically by creating a new parameter.
\attention Changes on the parameter panel won't take effect until you chose the button ``Save To DB''.

\marginlabel{Edit Parameter}
If you want to edit the information of a parameter, first chose the solver whose parameters you want to edit from the solver table. Then coose the parameter you like to edit and modify the information in the text fields below the table. Click ``Save To DB'' to persist your changes in the database.

\marginlabel{Delete Parameter}
To delete parameters of a solver, choose the solver and the parameter you want to delete (by holding \verb|Ctrl| in the parameter table, you can select multiple parameters). Click on ``Delete'' in the parameter panel.
\attention The delete action is performed immediately on the database! All your changes will be lost!

\marginlabel{Save changes to DB}
Adding and Editing solver, binary or parameter information will take effect to the database by choosing the button ``Save To DB''.

\marginlabel{Export}
To export the solver code and the binaries of a solver, choose the solver from the solver table and click on ``Export''. After selecting the directory where the exported files should be saved, EDACC creates a bunch of zip files with the exported code and binaries from the database.

\marginlabel{Reload from DB}
If you like to undo your changes you haven't already commited to the database by choosing ``Save To DB'', you can click on ``Reload from DB''. This has the effect that all information in the program will be stashed and reloaded from the database, so your uncommited changes will be lost.
% !TeX root = managedbmode.tex
\subsubsection{Instances}
\marginlabel{Instance} An instance is a practical instantiation of a problem. The instances tab provides functions for the user to add, remove, generate and organize instances.

\marginlabel{Instance Class} Instance classes enables the user to group and organize instances into different categories. It's possible that an instance is assigned to several instance classes. An instance classes can include other instance classes and are represented as an tree.

\marginlabel{Add Instance} To add one or more instances, use the button Add. In the following dialogue are four possible choices. 
\begin{enumerate}
	\item If automatic class generation is selected, the added instances are added to instance classes which
	are generated from the dependent on the directory of the instances to add. .
	
	\item If the automatic class generation isn't selected, the user have to choose an instance class from the 		instance class table of the dialogue. Else if automatic class generation is selected, the choice of an 			instance class is optional.
	
	\item To save the instances as compressed files in the database, select Compress.
	
	\item In the field File Extension, the user has to define the file extension of the instance files to add.
\end{enumerate}

To continue the process, the user has to use the button Ok and select the directory of the instances or their explicit files. This depends on the decisions made in the previous dialogue.
 

\marginlabel{Remove Instance}

\marginlabel{Generate Instance}

\marginlabel{Export Instance}

\marginlabel{Compute Property} 
% !TeX root = user_guide.tex
\subsection{Experiment Mode}
\subsubsection{Experiments}
\marginlabel{Experiment}\index{Experiment}\index{create/remove/edit experiments}
An experiment consists of solver configurations, instances and the number of runs for each solver configuration and instance. In the experiment tab the user can create/remove/edit experiments.

\marginlabel{Create}
By using the create-button in the first tab of the experiment mode an experiment can be created. This will open a dialog where you have to provide some data.
\begin{enumerate}
\item Name: the name for the new experiment
\item Description: a description for the experiment. Provide some useful information about the experiment to quickly identify experiments in the experiments table.
\end{enumerate}
After pressing the create-button the newly created experiment will be loaded automatically.

\marginlabel{Remove}
To remove an experiment use the appropriate button.

\marginlabel{Edit}
To edit an experiment use the appropriate button. There you can edit the data you provided by creating the experiment. If you want to change the priority of an experiment you can do this by directly editing this property in the experiment table. The same applies to activating and deactivating experiments. For more details about the effect of the priority property, see section \ref{sec:experiment_prioritization}. Deactivated experiments won't be computed by clients.

\marginlabel{Discard}
To discard an experiment use the appropriate button. This button is only available if an experiment is loaded.

\marginlabel{Load}
To load an experiment use the appropriate button or double click the experiment you want to load in the experiment table.

\marginlabel{Import}\label{sec:import_data_from_experiments}
It is possible to import data from other experiments. To import data from other experiments the following steps have to be applied:
\begin{enumerate}
\item Load the experiment you want to import data to
\item Press the import button in the experiment tab. This will open a new window with three tables for experiments, solver configurations and instances.
\item Select the experiments you want to import data from. This will update the solver configuration and instance tables to show all solver configurations and instances for the selected experiments. Orange rows mean that the solver configuration or instance in that row exists in the currently loaded experiment. Two solver configurations are considered as equal if they have the same solver binary associated and have the same launch parameters.
\item Select the solver configurations and instances to import
\item Select \textit{import finished jobs} if you also want to import jobs
\item Press \textit{Import} 

\attention Note that this action might generate new jobs. This \textit{might} happen if you import solver configurations and instances with their jobs to an experiment where some of the solver configurations and instances actually exist and they are in the \textit{same seed group}.
\end{enumerate}
\subsubsection{Client Browser}
The client browser represents all clients currently connected to the database. Red rows denote dead clients. \marginlabel{Dead clients}A client is considered as dead if the client didn't communicate with the database for a period of time.

The client browser also deals as the only way to directly communicate with clients. 

\marginlabel{Kill clients}After selecting the clients you can open the context menu with the right mouse button and select \textit{Kill Clients Hard} or \textit{Kill Clients Soft}. Hard means that the clients will terminate all currently computing jobs and sign off. Soft means that the clients won't start new jobs and will wait for the currently computing jobs to finish.

\marginlabel{Client details}To view the jobs which a client has computed in his lifetime you can double click a client entry in the client table. This will show a dialog with a table containing all jobs the client calculated and is currently calculating. You can also send messages to the clients in this dialog.
\subsubsection{Solvers}
\index{Solver configuration}Creating solver configurations is done in the solvers tab. This tab contains two tables on the right side and a panel with all solver configurations currently associated with this experiment.
\marginlabel{Choosing solvers}To create solver configurations you have to choose solvers for which you want create solver configurations. This can be done in the first table, the solvers table. By selecting some solvers and finally pressing the \textit{choose}-button, solver configuration prototypes will be created for the solvers. You can see the newly created solver configurations in the panel in the left side. This panel is organized as follows. For each solver exists one layer. Each layer contains all solver configuration for the associated solver. A solver configuration is titled with a name. This name can be changed and is used in the other areas of the GUI to identify a solver configuration. So it might be good practice to choose different names for the solver configurations in an experiment.

\marginlabel{Modifying solver configurations}A solver configuration consists of a solver binary, parameters and a seed group. The solver binary is chosen in the first combo box. The parameters can be specified in the parameters table. Just select the parameters you want for this solver configuration and specify their values if the have some. Finally you have to specify the seed group. The default seed group is \textit{0}. You might want to change that. See section \ref{section:seed_groups} for more information about seed groups.

\marginlabel{Importing solver configurations}To import solver configurations from other experiments you can import them in the experiments tab (see section \ref{sec:import_data_from_experiments}) or if you just want to import a solver configuration without jobs for a solver, you can select the solver in the solver table which will show all solver configurations in the database for that solver in the solver configuration table. Simply select the solver configurations you want to import and press the \textit{choose}-button.

\marginlabel{Tabular view for solver configurations}To change the view of the solver configuration panel to a tabular view, press the \textit{Change View}-button. This will change the panel into a table. Here you can remove multiple solver configurations by selecting them and opening the context menu by pressing the right mouse button and choose \textit{Remove}. It is also possible to edit solver configurations in that view by double clicking a solver configuration or by using the context menu.

\attention All modifications to solver configurations are not directly saved in the database. You can use the \textit{Undo}-button the undo all changes and load the last save state. By pressing the save button all modified and new solver configurations will be saved to and deleted solver configurations will be removed from the database.

\attention Modifying and saving solver configurations which have calculated runs might be not a good idea. Therefore the GUI supplies a possibility to reset the affected jobs. This might not be needed if the changed parameters have no effects to the results.
\subsubsection{Instances}

\subsubsection{Generate Jobs}

\subsubsection{Job Browser}

\subsubsection{Analysis}
\subsection{Property}
The management of result and instance properties is located at the menu bar, below the menu ``Property''. There are two menu items, called ``Import from CSV'' and ``Manage Properties''.

\subsubsection{Import from CSV}
After choosing the menu item ``Import from CSV'' , a file chooser opens and the \textbf{user} has to select the CSV file to import. The next displayed dialogue is seperated into two  different tables:
\begin{enumerate}
	\item CSV Property: The name of properties found in the CSV file. The names are identified from the first line of the choosen file.
	\item \edacc Property: A list of properties, avaible in the system. The user has to link the CSV properties with avaible system properties by using the checkboxes. 
\end{enumerate}

\marginlabel{Import CSV data}
After linking the CSV and \edacc properties, the \textbf{user} can import the data from the CSV file to the system using the button ``Import''. If existing data in the system should be replaced by the new imported data, the \textbf{user} has to choose ``Overwrite property data''.
\attention The data of a CSV property with no link to an existing property will not be added to the System. The \textbf{user} can also drop a CSV property by selecting one and using the button ``Drop''.

\marginlabel{Manage \edacc properties}
By pressing the button ``Manage'', located below the System property table, the Manage Properties dialogue, descriped in \ref{mangageProperties} ``Manage properties'', are displayed.

\subsubsection{Manage properties} \label{mangageProperties} 
This dialogue provides  the creation, removal and modification of properties to the \textbf{user}. The dialouge is structured into two parts:
\begin{enumerate}
	\item Property overview: A table that displays all avaible result and instance properties.
	\item Property details: Some input fields, showing detailed information of the selected property to the \textbf{user}, for example ``Property name'' or ``Description'. These input fields are also used during the creation of new properties. 
\end{enumerate}

\marginlabel{Create property}
By using the button ``New'' a new property is created. The button is located at the bottom of the dialogue. The new property  is defined by the following values, which have to be specified by the \textbf{user}:
\begin{enumerate}
	\item Property type: Two different types of properties are defined in \edacc, instance and result properties. 
	\item Name: The name of the new property, like ``Number of variables'' for a instance property.
	\item Description: An optional field, to specify the property and it's function.
	\item Property source: The choice of sources depends on the chosen type of the property. If instance property is selected, the \textbf{user} can choose between ``Instance'' (The instance file), ``InstanceName'' (Name of the Instance), ``ExperimentResult'' (The results from a calculated Experiment) and ``CSVImport'' (Only imported values, no calculation possible). For result properties, the \textbf{user} can choose between the four different outputs of an experiment, the ``Launcher''-, ``Solver''-, ``Verifier''- and ``WatcherOutput''. The property source defines the data resource from which the property values are calculated. 
	\item Calculation types: There are two possibilities to calculate a property, using an external script, program or via a regular expression. To use regular expressions, select ``Regular Expression'' and define one or more regular expressions into the textfield on the leftside of the selection button. Are more than one regular expressions used, the \textbf{user} has to seperate them with a new line. Or if the \textbf{user} wants to use an external script, he has to select ``Computation Method'', choose the computation method and define some parameters for the execution of the external script. \attention The defintion of parameters is optional.
	\item Value type: Choose the data type of the caluclated property values to afford their processing and displaying in the GUI. \edacc provides four default value types, ``String'', ``Float'', ``Integer'' and ``Long''. The \textbf{user} can expand the list of value types by adding new value types. This process is  explained at \ref{definePropertyValuetype} ``Define property value type''. 
	\item Multiple occurrences: With this option the \textbf{user} can specify if the property occurres single or multiple times in a single property source object. 
\end{enumerate} 
\attention The new property is not saved until the button ``Save' is used. If the \textbf{user} selects a property or use the button ``New'' at the bottom side of the dialogue, the input in the fields are deleted. 

\marginlabel{Remove property}
The \textbf{user} can remove an existing property from \edacc, by  selecting the property and use of the button ``Remove''.

\marginlabel{Import property}
Properties exported with the GUI can be imported via the button ``Import'' , located at the bottom of the dialogue. The \textbf{user} has to select the file to import with the displayed file chooser. This feature, combined with the export function of properties, allows \textbf{users} to share properties.

\marginlabel{Export property}
Allows the \textbf{user} to export properties to other \edacc systems.

\marginlabel{Define property value type}\label{definePropertyValuetype}
To create new value types of the property values, the button ``New'' has to be used. The shown dialogue enables two functions to the \textbf{user}:
\begin{enumerate}
	\item Add: By selecting the jar archive, containing implementations of the java interface class ``PropertyValueType'', new value types can be added to the \edacc system. The \textbf{user} has to select the corresponding java classes of the value types from the lis, displaying all found classes of the jar archive.
	\item Remove: Only \textbf{user} defined value types can be deleted via the ``Remove'' button. Value types declared default cannot be removed.
\end{enumerate}

\marginlabel{Add computation Method}
After using the button ``New'' on the left side of the label ``Computation method'', a dialogue divided into a table, containing all avaible computation methods, and a form for a detailed view of the computation methods is shown. To add a computation method, the user has to use the button ``New'', below the overview table and fill in the three input fields:
\begin{enumerate}
	\item Name: Defines the name of the new computation method.
	\item Description: It is an optional field to comment and specifiy the computation method.
	\item Binary: The \textbf{user} has to choose the binary of the computation method via a file chooser.
\end{enumerate}
\attention The input of the new computation method is not saved until the button ``Save'' is pressed. 

\attention The external script or program of the computation method recieves the data to process via standard input  and has to commit the results via standard output. \edacc will call the computation method with only a single object, like an instance file, terminate and restart it with the next source object.

\newpage

\section{Parameter search space specification}
\label{parameter_spec}
\subsection{Definitions}
A parameter is an input variable of a program and is defined by a name, a domain, a prefix (which can also be empty) and an order number.
Additionally there are two booleans. ''space'' that indicates whether to put a space between the prefix and the value. ''attach to previous''
indicates whether there should be a space between the previous parameter (according to order) and this one.
''attach to previous'' can be useful for parameters that look like ''-prefix v1[,v2,v3]''. '',v2'' and '',v3'' can be modeled as parameters that attach
to the preceding ''-prefix'' parameter.

\begin{definition}
A solver configuration is a list of parameters and their assigned values.
\end{definition}

\begin{definition}
A domain defines the set of possible values that can be assigned to a parameter (in a solver configuration). It can be one of the following or the union of any number of them (except for the flag domain, which can only occur on its own).
\begin{enumerate}
\item real: values between a lower and an upper bound
\item integer: values between a lower and an upper bound
\item ordinal: list of values in a min to max order
\item categorial: set of possible values
\item optional: consists only of a special value "not specified"
\item flag: consists of two special values "on" and "off" (for parameters that are flags, i.e. present or not)
\end{enumerate}
\end{definition}

\begin{definition}
The parameter space of a solver is defined by its parameters and their possible values. The parameter space can be further constrained by
dependencies between parameters such as
\begin{enumerate}
\item Parameter X can be specified if parameter Y takes on certain values
\item Parameter X has to be specified if parameter Y takes on certain values
\item Parameter X has to take on certain values depending on the values of parameters Y, Z, ...
\end{enumerate}
\end{definition}

There are several tasks that come up in the context of EDACC: Determine if a given solver configuration is valid, i.e. in the solver's parameter space.
Given the parameter space, construct a valid solver configuration. Given a valid solver configuration, find a ''neighbouring'' solver configuration that is also valid.

\begin{definition}
A parameter graph is a directed, acyclic graph that represents the parameter space. It consists of AND-Nodes and OR-Nodes and edges between them. Edges are directed and allowed only
between different types of nodes. OR-Nodes can have multiple incoming edges, while AND-Nodes can only have exactly one incoming edge. Additionally edges have a group number which is 0 if the edge doesn't belong to any group.
Parameter graphs have a single unique AND-Node without any incoming edges. This node will be referred to as start node.
\end{definition}

\begin{definition}
OR-Nodes have a reference to a parameter.
\end{definition}

\begin{definition}
AND-Nodes have a domain and a parameter reference to the same parameter as the preceding OR-node.
AND-Nodes partition the possible values of the parameter that they (and the preceding OR-node) reference.
The domain of an AND-Node has to be a subset of the domain of the preceding OR-Node.
\end{definition}

The general idea is that the parameter space is specified by following the structure of the graph from the start node and constraining the parameters using the domains encountered
on the nodes.
AND-Nodes imply that all outgoing edges have to be followed while OR-Nodes mean that exactly one edge has to be followed.

More formally:
\begin{definition}
A solver configuration is valid if the start node (an AND-Node) of the parameter graph is satisfied. Satisfied means:
\begin{enumerate}
\item an AND-Node is satisfied if the corresponding parameter value lies in its domain and all OR-nodes adjacent via ungrouped edges are satisfied.
\item an OR-Node is satisfied if exactly one adjacent AND-Node is satisfied and for at least one set of incoming edges with common group number the preceding AND-Nodes are all satisfied.
\end{enumerate}
\end{definition}
\newpage

\subsection{Algorithms on parameter graphs}

Constructing a valid, random solver configuration:
\begin{verbatim}
Input: parameter graph; Output: (random) solver configuration
done_and := {startnode} # set containing only the start node
done_or := {} # empty set

L := list of OR-nodes adjacent to startnode

while (L has nodes with >= 1 group of edges coming from \
       AND-nodes in done_and):
    or_node := choose and remove such a node randomly from L
    done_or.add(or_node)
    and_node := choose an AND-node adjacent to or_node
                randomly (or with user input)
    done_and.add(and_node)
    Assign a random (or user chosen) value from and_node.domain to
     the and_node.parameter of the solver configuration 
    
    Add all OR-nodes that are adjacent to and_node, not yet in
        L and not in done_or to L

return the solver configuration
\end{verbatim}

Validating a solver configuration (TODO: issues with optional parameters):
\begin{verbatim}
Input: Parameter Graph, Solver configuration; Output: Boolean
assigned_and_nodes := Set of AND-nodes with parameters that are
                        part of the solver config

Test if all required OR-node parameters are present
 i.e. of OR-nodes that have at least one incoming edge
 group from assigned AND nodes
 
or_nodes := Set of preceding OR-nodes of assigned_and_nodes
for each or_node in or_nodes:
    - Test if or_node has exactly one adjacent AND-node in
      assigned_and_nodes
    - Test if at least one group of edges comes from
      assigned AND-nodes

Test if all OR-nodes adjacent to start node are in or_nodes

return True if all tests were passed
\end{verbatim}

Open problems:
\begin{itemize}
\item Discretise real valued parameters. Especially the discretisation of domains divided into several domains (several AND-Nodes) is unclear.
\item find an algorithm to iterate over the neighbourhood of a given solver configuration (with discretised parameters)
\item crossover and mutation operators for genetic algorithms
\end{itemize}

\newpage

\subsection{Example}
\marginlabel{\Eexample}
Consider a solver that has the following parameters:
\begin{itemize}
\item \textit{c1} which takes on integer values in $[1,10]$.
\item \textit{ps} which takes on real values in $[0,1]$.
\item A flag called \textit{lookahead} which can be present or not.
\item A categorical parameter \textit{steps} which takes on values in $\{0,1,2,3,4\}$.
\item Another categorical parameter \textit{method} whose value is either ''hybrid'' or ''atom''.
\item A parameter \textit{prob} which can be left out or take on real values in $[0,1]$.
\end{itemize}
Furthermore there are some restrictions and requirements:
\begin{itemize}
\item Both \textit{c1} and \textit{ps} have to be always specified.
\item If the \textit{lookahead} flag is present, both \textit{steps} and \textit{method} have to be present.
\item If \textit{steps} takes on the value $0$ and \textit{method} takes on the value ''hybrid'', then the parameter \textit{prob} can take on values in its real domain $[0,1]$ or be left out.
\end{itemize}
This parameter space can be encoded in a parameter graph as defined earlier in the following way:
\begin{figure}[htb]
\includegraphics[width=10cm]{paramgraph.pdf}
\end{figure}

The two red edges imply the membership of the edges to the same edge group $\neq 0$. Black edges mean that the edge doesn't belong to any group.
For simplicity, the parameter references of AND-Nodes (to the same parameter as the preceeding OR-Node) are not shown in the graph.
\clearpage
\subsection{Implementation etc.}
\begin{itemize}
\item define parameter graphs in XML / GUI 
\item API documentation for configurators
\end{itemize}
\clearpage



\newpage
\section{Client}
\label{client}

\subsection{Introduction}
The computation is client is used to compute the jobs of experiments. Usually there have to be a lot of jobs computed to evaluate experiments
and since they are independent from each other, this task can be parallelized across many CPU cores. The computation client can be started on arbitrarily many
machines and will manage the available CPUs and start jobs from the available experiments. It connects to the central database and downloads all required resources
such as instances and solver binaries and writes back the results to the database.

\subsection{System requirements}
\label{clientSR}
\index{System requirements client}
The client is written in C/C++ and should be able to run on most Linux distributions where a MySQL C connector library is available.

Because the central storage location for all required computational ressources, experiment metadata and results is a MySQL database, the client has to be able to establish a connection to the machine that hosts the database. 
\marginlabel{TCP/IP Connections:}
This means that the machines where the client runs on have to be able to establish a TCP/IP connection
to the database machine. 

The client was mainly tested on the bwGRID\footnote{{http://www.bw-grid.de/}}, a distributed computer cluster that consists of several hundred
nodes at several physical locations at universities of Baden-W�rttemberg, Germany. Even though the machine hardware is homogenous, the network topology of bwGRID is not.
\marginlabel{Connection alternatives:}
In cases where direct network access from the computation nodes back to the database server is not possible it is usually possible to tunnel a connection
over the cluster's login node back to the database via SSH. \todo provide example

Other than that, the client has to be able to write temporary files to some location on the filesystem. This can be configured (\ref{client_command_line_arguments}) if it differs from the client binary location.

Because the client will download missing solver binaries and instances and upload results it also needs a reasonably fast network connection to the database. 
\marginlabel{Shared filesystems:}
\index{Shared filesystems}
Shared filesystems can considerably reduce the required bandwidth since every file is only downloaded once. Alternatively you can create
a package from within the GUI that contains all solver binaries and instances of an experiment. However, if you modify experiments while the client is running it will still download missing files.

\newpage
\subsection{Usage}
\subsubsection{Configuration}
\marginlabel{Configuration file:}
\index{Client configuration}
Configuration is done by some command line arguments and a simple configuration file, called ''config''. This file has to be in the \textbf{working directory} of the client at runtime.
In the configuration file you have to specify the database connection details and which hardware the client runs on. This is done by configuring so called ''grid queues'' in the GUI application.
They contain some basic information about the computation hardware such as number of CPUs per machine. The client will then use this information to run as many parallel jobs as
the grid queue information allows it on each machine where it is launched. Here is a sample configuration file:
\marginlabel{\Eexample}
\begin{verbatim}
host = database.host.foo.com
port = 3306
username = dbusername
password = dbpassword
database = dbname
gridqueue = 3
verifier = ./verifiers/SAT
\end{verbatim}
\index{Grid queue}
Note that the gridqueue value is simply the ID of the grid queue. Another (optional) configuration option is the verifier line. It tells the client if it should run a program on the output that a
solver generated. For example, if your experiments consist of attempting to solve boolean propositional logic formulas you can use a SAT verifier that tests if the solution given by a solver
is actually correct. The verifier configuration option is simply a path to an executable that the client calls with standardized parameters. See section \ref{sec:verifiers} for more information on verifiers.


Beside the configuration file there are several command line options the client accepts, please also see ''./client --help'':
\marginlabel{Command line arguments:}
\label{client_command_line_arguments}
\begin{verbatim}
-v <verbosity>:
  Integer value between 0 and 4 (from lowest to
  highest verbosity)
-l:
  If flag is set, the log output is written to a file
  instead of stdout.
-w <wait for jobs time (s)>:
  How long the client should wait for jobs after
  it didn't get any new jobs before exiting.
-i <handle workers interval ms>:
  How long the client should wait after handling
  workers and before looking for a new job.
-k:
  Whether to keep the solver and watcher output files after
  uploading to the DB. Default behaviour is to delete them.
-b <path>:
  Base path for creating temporary directories and files.
-h:
  Toggles whether the client should continue to run even
  though the CPU hardware of the grid queue is not homogenous.
-s:
  Enables simulation mode where the client will fetch and run
  jobs but won't write any results back to the database.
\end{verbatim}

Verbosity controls the amount of log output the client generates. A value of 4 is only useful for debugging purposes, a value of 0 will make the client log important messages and all errors.

If the ''l'' flag is set, log output goes to a file whose name includes the hostname and IP address of the machine the client runs on. This is done to avoid name clashes in shared filesystems typically found in computer clusters.
Otherwise log output goes to standard output.

With the ''-w'' option you can tell the client how long to wait before exiting after it didn't start any jobs. This can be useful to keep the clients running and ready to process new jobs
while you evaluate preliminary results and add new jobs or whole experiments. The wait option is also used to determine how long attempts should be made to reconnect to the database after
connection losses. The default value is 10 seconds.

The ''-i'' option controls how long the client should wait between its main processing loop iterations. If this value is low, it will look for new jobs when there are unused CPUs more frequently.
For maximum job throughput this value should be lower than the average job processing time but lower values will also put more strain on the database and increase the client's CPU usage. The default
value is 100ms which should work fine in most cases. The client will also adapt to situations where there are free CPUs but no more jobs and increase the interval internally and fall back to the configured
value once it got another job.

The ''-k'' flag tells the client that it should keep temporary job output files after a job is finished. The default behaviour is to delete them.

The ''-b'' base path option can be used to specify a directory the client can use to write temporary files to. The default value is ''.'', i.e.
the working directory at runtime.

\marginlabel{inhomogenous machines}
\attention The first client to start with a particular grid queue will write the information about the machine it runs on to the grid queue entry in the database. All following clients will then
compare their machines to the information in the grid queue and exit, unless the number of cores and the CPU model name match. With the ''-h'' option you
can override this behaviour.

\subsubsection{Launching}
After configuration you can simply run the client on your computation machines. On computer clusters there are often queuing systems that you have to use to gain access to the nodes.
On bwGRID for example, we could use the following short PBS (portable batch script) and submit (\textit{qsub scriptname}) our client to a node with 8 cores:
\begin{verbatim}
#!/bin/sh
#PBS -l walltime=10:00:00
#PBS -l nodes=1:ppn=8
cd /path/to/shared/fs/with/client/executable
./client -v0 -l -i200 -w120
\end{verbatim}
\attention You should always run the client from within its directory (i.e. cd to the directory) to avoid problems with relative paths such as the verifier path from the example configuration above.

As soon as clients start you should be able to see jobs changing their status from ''not started'' to ''running'' in the GUI's or Web frontend's job browsers.

\subsubsection{Troubleshooting}
If errors or failures occur the client will always attempt to shut down cleanly, that is stop all running jobs and set their status to ''client crashed'' and write
the last lines of its log output as ''launcher output'' to each job. This can fail when network connections fail or the client receives a SIGKILL signal causing it to exit immediately.
In case of network failures you should still be able to find useful information in the client's logfile on the local filesystem.

\subsection{Verifiers}
\label{sec:verifiers}
Verifiers are programs that the client runs after a job finishes. Verifiers are getting passed the instance of the job and the solver output as arguments and are supposed to
write a newline character followed by a (textual/ASCII) integer result code at the end of their output. The result code should convey some information about the result of
the job, for example whether the output of the solver is correct given the problem instance.
This code will be written to the database as ''result code'' while the verifier's exit code will be written as ''verifier exit code''. Any output the verifier writes to standard out will
be written as ''verifier output''. The call specification for a verifier binary looks like this:
\begin{verbatim}
./verifier_binary <path_to_instance> <path_to_solver_output>
\end{verbatim}
We provide a verifier for the SAT problem that works on CNF instances in DIMACS format and solvers that adhere to a certain output format (see the source code).
If you want to write an own verifier specific to your problem you can also use the source code as implementation example.
\attention Note that you have to make sure
that your possible result codes are specified in the \textit{ResultCodes} table in the database before running clients or there will be errors when the client tries to write results. By convention,
the web frontend and GUI application consider status codes that begin with a decimal ''1'' as correct answers.

\subsection{Experiment priorization}
\label{sec:experiment_priorization}
Sometimes it can be useful to compute several experiments in parallel but give some a higher priority than others. In order to accomplish that, experiments can be marked as inactive and individual jobs can be prioritized. Only jobs of active experiments with priority equal to or greater than 0 are considered for processing
by the client. Futhermore, experiments can be assigned a priority. The clients will then try to match the relative number of CPUs working on an experiment with its relative priority to
all other experiments that are assigned to the same grid queue. For example, if you have three experiments with priorities 100, 200 and 300 respectively the running clients will try to have 16\% of CPUs working on the first, 33\% of CPUs working on the second and 50\% of CPUs working on the third experiment.

\clearpage

\attention The client is running solely on Unix and is not distributed yet for Windows systems.

\newpage
\section{Web Frontend}
\label{web_frontend}
\upshape
\index{Web Frontend}
\subsection{Introduction}
The Web Frontend provides access to experiment information and analysis tools in a read-only manner
and accessible by a web browser.

\subsection{System requirements}
\index{System requirements, Web Frontend}
The web frontend is implemented as Python WSGI web application and makes use of several libraries.
Since it interfaces with R to draw plots it also depends on R and a Python interface to R, which unfortunately
only works properly on Linux right now.
WSGI applications can be deployed on a variety of web servers or even run standalone on a web server that comes with the
Python standard library.
The following list contains all dependencies and prerequisites of the web frontend (see below for installation instructions).
\begin{itemize}
\item Python 2.6.5 or 2.7 http://www.python.org
\item R 2.11 (language for statistical computing and graphics)
\item R package 'np' (available via R's CRAN)
\item SQLAlchemy 0.6.5 (SQL Toolkit and Object Relational Mapper)
\item mysql-python 1.2.3c1 (Python MySQL adapter)
\item Flask 0.6 (Micro Webframework)
\item Flask-WTF 0.3.3 (Flask extension for WTForms)
\item Flask-Actions 0.5.2 (Flask extension)
\item Werkzeug 0.6.2 (Webframework, Flask dependency)
\item Jinja2 2.5 (Template Engine)
\item PyLZMA 0.4.2 (Python LZMA SDK bindings)
\item rpy2 2.1.4 (Python R interface)
\item PIL 1.1.7 (Python Imaging Library)
\item Numpy 1.5.1
\item pygame 1.9 (Graphics library)
\end{itemize}

\subsection{Installation}
\index{Installation, Web Frontend}
To get rpy2 working the GNU linker (ld) has to be able to find libR.so. Add the folder containing
libR.so (usually /usr/lib/R/lib) to the ld config: Create a file called R.conf containing the
path in the folder /etc/ld.so.conf.d/ and run ldconfig without parameters as root to update.
Additionally, you have to install the R package 'np' which provides non-parametric statistical
methods. This package can be installed by running "install.packages('np')" within the R interpreter (as root).

The following installation example outlines the step that have to be taken to install the web frontend on Ubuntu 10.04
running on the Apache 2.2.14 web server. For performance reasons (e.g. query latency) the web frontend should run on the
same machine that the EDACC database runs on.
\marginlabel{\Eexample}
\begin{enumerate}
\item Install Apache and the WSGI module: \begin{verbatim}apt-get install apache2 libapache2-mod-wsgi\end{verbatim}
\item{ Copy the web frontend files to /srv/edacc\_web/, create an empty error.log file and change their ownership to the Apache user: 
\begin{verbatim}
  touch /srv/edacc_web/error.log
  chown www-data:www-data -R /srv/edacc_web
\end{verbatim}
}
\item{ Create an Apache virtual host\\
(new file at /etc/apache2/sites-available/edacc\_web)
\begin{verbatim}
<VirtualHost *:80>
  ServerAdmin email@email.com
  ServerName foo.server.com

  LimitRequestLine 51200000

  WSGIDaemonProcess edacc processes=1 threads=15
  WSGIScriptAlias / /srv/edacc_web/edacc_web.wsgi

  Alias /static/ /srv/edacc_web/edacc/static/

  <Directory /srv/edacc_web>
    WSGIProcessGroup edacc
    WSGIApplicationGroup %{GLOBAL}
    Order deny,allow
    Allow from all
  </Directory>

  <Directory /srv/edacc_web/edacc/static>
    Order allow,deny
    Allow from all
  </Directory>
</VirtualHost>
\end{verbatim}
}
\item{Install dependencies and create a virtual environment for Python libraries:
\begin{verbatim}
apt-get install python-pip python-virtualenv python-scipy python-pygame python-imaging
virtualenv /srv/edacc_web/env
apt-get build-dep python-mysqldb
apt-get install r-base
echo "/usr/lib/R/lib" > /etc/ld.so.conf.d/R.config
ldconfig
source /srv/edacc_web/env/bin/activate
pip install mysql-python
pip install rpy2
pip install flask flask-wtf flask-actions
pip install sqlalchemy pylzma numpy
\end{verbatim}
}
\item{Install R libraries (''R'' launches the R interpreter):
\begin{verbatim}
R
install.packages('np')
\end{verbatim}
}
\item{Create a WSGI file at /srv/edacc\_web/edacc\_web.wsgi with the following contents:
\begin{verbatim}
import site, sys, os
site.addsitedir('/srv/edacc_web/env/lib/python2.6/site-packages')
sys.path.append('/srv/edacc_web')
sys.path.append('/srv/edacc_web/edacc')
os.environ['PYTHON_EGG_CACHE'] = '/tmp'
sys.stdout = sys.stderr
from edacc.web import app as application
\end{verbatim}
}
\item Configure the web frontend by editing /srv/edacc\_web/edacc/config.py, see~\ref{wf:configuration} for details.
\item{Enable the Apache virtual host created earlier:
\begin{verbatim}
a2ensite edacc_web
service apache2 restart
\end{verbatim}
}
\item The web frontend should now be running under http://foo.server.com/
\end{enumerate}

\subsection{Configuration}
\label{wf:configuration}
\index{Configuration, Web Frontend}
All configuration is done in a Python file located at ''edacc/config.py''. The options are documented in the sample configuration
file which is included in the distribution package.\attention Most importantly, you should disable debugging mode when making the Web Frontend
accessible from the network to avoid security problems. At the end of the file you can configure the database connection and the list
of EDACC databases that should be made available by the Web Frontend.

\subsection{Feature overview}
\clearpage

\section{Monitor}

\section{Troubleshooting}



\begin{fullpage}
\stepcounter{section}
\addcontentsline{toc}{section}{\numberline {\thesection} Glossar}
\printglossary
\end{fullpage}

\printindex

\end{document}
